\documentclass{article}
\usepackage[a4paper, inner=1.7cm, outer=2.7cm, top=2cm, bottom=2cm, bindingoffset=1.2cm]{geometry} 
\usepackage[english]{babel}
\usepackage{blindtext}
\usepackage{graphicx}
\usepackage{wrapfig}
\usepackage{enumitem}
\usepackage{amsmath}
\usepackage{index}
\usepackage{import}
\usepackage{epigraph}
\makeindex

%new commands: 
\newcommand{\norm}[1]{\left\lVert#1\right\rVert}
\newcommand{\innerproduct}[1]{\langle#1\rangle}

\newtheorem{theorem}{Theorem}[section]
\newtheorem{corollary}{Corollary}[theorem]
\newtheorem{lemma}[theorem]{Lemma}

\begin{document}
\title{\Large{\textbf{Linear Algebra: Map of theorems}}}
\author{Tan Wee Han}
\maketitle

\let\cleardoublepage\clearpage
\tableofcontents

\pagestyle{plain} %remove page headers but keep page numbers

\makeatletter
%\setlength{\@fptop}{0pt} % to put the figure on top of the page.
\makeatother


\section{Bilinear Forms and Inner Product}
Cauchy-Schwarz Inequality: $\norm{\innerproduct{u,v}} \leq \norm{u}\norm{v}$ \\
Triangle Inequality: $\norm{u+v} \leq \norm{u} + \norm{v}$ \\
Pythagoras Theorem: $\norm{u+v}^2 = \norm{u}^2 + \norm{v}^2$ \bigskip


\noindent 

\begin{theorem}
    Let $f$ be a function whose derivative exists in every point, then $f$ is 
    a continuous function.
\end{theorem}

\section{Test}

\begin{theorem}
    Let $f$ be a function whose derivative exists in every point, then $f$ is 
    a continuous function.
\end{theorem}




\end{document}
