\documentclass{article}
\usepackage[a4paper, inner=1.7cm, outer=2.7cm, top=2cm, bottom=2cm, bindingoffset=1.2cm]{geometry} 
\usepackage[english]{babel}
\usepackage{blindtext}
\usepackage{graphicx}
\usepackage{wrapfig}
\usepackage{enumitem}
\usepackage{amsmath}
\usepackage{amsfonts} %for using mathbb
\usepackage{index}
\usepackage{import}

\usepackage[bookmarks=true]{hyperref}
\usepackage{bookmark}

\makeindex

%new commands: 
\newcommand{\norm}[1]{\left\lVert#1\right\rVert}
\newcommand{\innerproduct}[1]{\langle#1\rangle}

\newtheorem{theorem}{Theorem}[section]
\newtheorem{definition}{Definition}[section]
\newtheorem{proposition}{Proposition}[section]
\newtheorem{corollary}{Corollary}[theorem]
\newtheorem{lemma}[theorem]{Lemma}

\numberwithin{theorem}{subsection} % important bit
\numberwithin{definition}{subsection} % important bit
\numberwithin{proposition}{subsection} % important bit

\begin{document}
\title{\Large{\textbf{Linear Algebra: Map of theorems}}}
\author{Tan Wee Han}
\maketitle

\let\cleardoublepage\clearpage
\tableofcontents

\pagestyle{plain} %remove page headers but keep page numbers

\makeatletter
%\setlength{\@fptop}{0pt} % to put the figure on top of the page.
\makeatother


\section{Inner Product Introduction}


\subsection{Vector inequalities}
Cauchy-Schwarz Inequality: $\norm{\innerproduct{u,v}} \leq \norm{u}\norm{v}$ \\
Triangle Inequality: $\norm{u+v} \leq \norm{u} + \norm{v}$ \\
Pythagoras Theorem: $\norm{u+v}^2 = \norm{u}^2 + \norm{v}^2$ \bigskip

\subsection{Orthogonal and Orthonormal basis}
\begin{theorem}
    An orthogonal set of nonzero vectors is linearly independent.
\end{theorem}

\begin{corollary}
    If $V$ is a finite dimensional inner product space and $n=dim V$, then any orthogonal set of nonzero vectors in $V$ is finite and contains at most n vectors.
\end{corollary}

\begin{lemma}
    Let $ \mathcal{B} = \{v_1,...,v_n\}$ be an orthonormal basis of $V$. Then for any $v \in V,$ \\
    \begin{equation*}
        V = \innerproduct{v,v_1}v_1 + \innerproduct{v,v_2}v_2 +...+
        \innerproduct{v,v_n}v_n
    \end{equation*}
\end{lemma}

\begin{corollary}

    Let $ \mathcal{B} = \{v_1,...,v_n\}$ be an orthonormal basis of $V$,
    $T:V\rightarrow V$ a linear operator and $[T]_\mathcal{B} = (a_{ij})$ Then for all $i,j$ \\
    \begin{equation*}
    a_{ij}=\innerproduct{T(v_j), v_i}
    \end{equation*}
\end{corollary}

\begin{theorem}
    Every finite dimensional inner product space has an orthonormal basis.
\end{theorem}



\subsection{Orthogonal complement and projection}
\begin{theorem}
    If $W$ is a subspace of an inner product space V, then its orthogonal complement
    $W^{\perp}$ is a subspace of $V$. In addition, we have
    \begin{equation*}
        W \cap W^{\perp} = \{\mathbf{0}\}
    \end{equation*}
\end{theorem}

\begin{theorem}
    If $W$ is a finite dimensional subspace of an inner product space V, then
    \begin{equation*}
        V = W \oplus W^{\perp} 
    \end{equation*}
\end{theorem}

\begin{theorem}
    If $\{w_1, ..., w_k\}$ is an orthonormal basis of $W$ then 
    \begin{equation*}
        \mathbf{proj}_W(v) = \Sigma^k_{j=1} \innerproduct{v,w_j}w_j
    \end{equation*}
\end{theorem}


\begin{theorem} 
    \textbf{Best Approximation:}
    If $W$ a finite dimensional subspace of an inner product space $V$ and $v \in V$, then
    \begin{equation*}
        \norm{v-\mathbf{proj}_W(v)} < \norm{v-w}
    \end{equation*}
    for every vector $w$ in $W$ different from $\mathbf{proj}_W(v)$.
\end{theorem}



\begin{theorem}
    \textbf{Least Square Solution:}
    For any real linear system $A\mathbf{x} = \mathbf{b}$ the associated normal system 
    \begin{equation*}
        (A^tA)\mathbf{x} = A^t \mathbf{b}
    \end{equation*}
is consistent, and all its solutions are least square solutions of $A\mathbf{x} = \mathbf{b}$
\end{theorem}


\subsection{Adjoint of linear operator}

\begin{theorem}
    Let $V$ be a finite dimensional inner product space over $\mathbb{F}$. If $f:V \rightarrow \mathbb{F}$ is a linear functional, then there exists a unique vector $ u \in V$ such that. 
    \begin{equation*}
        f(v) = \innerproduct{v,u} \qquad \text{for all } v \in V
    \end{equation*}
\end{theorem}

\begin{theorem}
    Let $T: V \rightarrow V$ be a linear operator on a finite dimensional inner product
    space $V$. Then there exists a unique linear operator $T^{*}: V \rightarrow V$ such
    that 
        \begin{equation*}
            \innerproduct{T(u),v} = \innerproduct{u,T^*(v)}
        \end{equation*}
    $T^*$ is called the adjoint of T.
\end{theorem}

\begin{theorem}
    Let $T:V \rightarrow V$ a linear operator on a finite dimensional inner product space
    V and B be an orthonormal basis of V. Then 
        \begin{equation*}
            [T^*]_\mathcal{B} = ([T]_\mathcal{B})^*
        \end{equation*}
        is the conjugate transpose of the matrix $[T]_\mathcal{B}$
\end{theorem}

\begin{theorem}
    Let $T,T_1 \text{ and } T_2$ be linear operators on a finite dimensional inner product
    space V. Then: 
    \begin{flalign*}
        &(i) \quad (T_1 + T_2)^* = T_1^* + T_2^* \\
        &(ii) \quad c(T)^* = \overline{c} T^* \\
        &(iii) \quad (T_1T_2)^* = T_2^*T_1^* \\
        &(iv) \quad (T^*)^* = T \\
    \end{flalign*}
\end{theorem}

\subsection{Self-adjoint and Normal Operator}
Main motivation for this section is to find out under what conditions does $V$ have an
orthonomal basis of eigenvectors for $T$.

\begin{definition}
    A linear operator $T$ on a finite dimensional inner product space V is called
    $\textbf{self-adjoint}$ if $T=T^*$, i.e. it satisfies the equation:
    \begin{equation*}
        \innerproduct{T(u),v} = \innerproduct{u,T(v)}
    \end{equation*}
\end{definition}

\begin{theorem}\label{selfadjointrealandorthogonal}
    (p. 312)
    A self-adjoint linear operator $T$ on a finite dimensional inner product space V.Then
    all eigenvalues of $T$ is real. And the eigenvectors associated with the distinct
    eigenvalues are orthogonal.
\end{theorem}

\begin{theorem}
    (p. 313)
    On a finite dimensional inner product space of positive dimension, every self-adjoint
    operator has at least one eigenvalue which also means it has at least one eigenvector.\\

    (note proof of this in complex vector space do not require $T$ to be self-adjoint, but in
    real vector space it does)
\end{theorem}

\begin{lemma}\label{lem1}
    Let $V$ be a finite dimensional inner product space. Let $T:V
    \rightarrow V$ be any linear operator. Suppose $W$ is a subspace of $V$ which is
    invariant under T. Then $W^\perp$ is invariant under $T^*$
\end{lemma}

\begin{theorem}\label{orthgonallydiagonalizable}
    Let $V$ be a finite dimensional inner product space (complex or real). Let $T:V
    \rightarrow V$ be a self adjoint linear operator. $T$ is orthgonally diagonalizable if
    and only if it is self-adjoint. \\
    (proof uses Lemma \ref{lem1})
\end{theorem}

\begin{corollary}
    Let $A$ be an $n \times n$ Hermitian matrix. Then there is a unitary matrix $P$ such
    that $P^{-1}AP$ is diagonal. (i.e. $A$ is unitariliy equivalent to a diagonal matrix).
    If $A$ is a real symmetric matrix, there is a real orthogonal matrix $P$ such that 
    that $P^{-1}AP$ is diagonal.
\end{corollary}

\begin{definition}
    Let $V$ be a finite dimensional inner product space. Let $T:V \rightarrow V$ be
    a linear operator on V. We say that $T$ is normal if it commutes with its adjoint: 
    \begin{equation*}
        TT^* = T^*T
    \end{equation*}
\end{definition}

\begin{lemma}\label{lem2}
    Let V be a finite dimensional complex inner product space and $T:V\rightarrow V$
    a normal linear operator. Then if $v \in V$ is an eigenvector of $T$ corresponding to
    the eigenvalue $\lambda$, then it is also an eigenvector of $T^*$ corresponding to the
    eigenvalue $\overline{\lambda}$. That is for $v \in V$,
    \begin{equation*}
        T(v) = \lambda v \Rightarrow T^*(v) = \overline{\lambda} v
    \end{equation*}
\end{lemma}

\begin{lemma}\label{lem3}
    Let V be a finite dimensional complex inner product space and $T:V\rightarrow V$
    any linear operator. Then $V$ has an orthonormal basis $\mathcal{B}$ such that matrix
    $[T]_\mathcal{B}$ is upper triangular. 
\end{lemma}

\begin{theorem}
    A linear operator on a finite dimensional complex inner product space is orthogonally
    diagonalizable if and only if it is normal. \\
    (proof uses lemma \ref{lem2} and \ref{lem3})
\end{theorem}
        

\subsection{Unitary Operator}

\begin{theorem}
    Let $V$ and $W$ be a finite dimensional inner product spaces over the same field.  If
    $f : V \rightarrow W$ is a linear transformation, the following are equivalent:

    \begin{enumerate}
        \item $\innerproduct{T(u),T(v)} = \innerproduct{u,v}$ for all $u,v \in V$. i.e. $T$ preserves inner product spaces
        \item $T$ is an inner product space isomorphism
        \item $T$ carries every orthonormal basis for $V$ onto an orthonormal basis for $W$
        \item $\norm{T(v)}=\norm{v}$ for all $v \in V$.
    \end{enumerate}
\end{theorem}

\begin{definition}
    A unitary operator on an inner product space is an isomorphism of the space onto
    itself.
\end{definition}

\begin{theorem}
    Let $T$ be a linear operator on an inner product space V. Then $T$ is unitary if and
    only if the adjoint $T^*$ of  T exists and $TT^*=T^*T = I$
\end{theorem}

\begin{theorem}
    Let $V$ be a finite dimensional inner product space and let $T: V \rightarrow V$ be
    a linear operator. Then $T$ is unitary if and only if the matrix of $T$ in some (or every) ordered basis is
a unitary matrix. 
\end{theorem}

\begin{theorem}
    (p.305)
    For every invertible complex $n \times n$ matrix $B$, there exists a unique lower-triangular
    matrix M with positive entries on the main diagonal such that $MB$ is unitary.
\end{theorem}

\begin{corollary}
    (p.307)
    Let $T^+(n)$ be the set of all complex $n \times n$ lower triangular matrices with
    positive entries on the main diagonal. Let $U(n)$ be a group of unitary matrices. For each $B$ in $GL(n)$ there exists unique matrices $N$ and $U$ such that $N \in
    T^+(n)$ and $U \in U(n)$.
\end{corollary}


\section{Operators on Inner Product Spaces}
Let $\mathcal{L}(V;V)$ denote the space of all linear operators from V to V. \\
Let $\mathcal{F}(V,V,\mathbb{F})$ denote the space of all forms on V. 

\subsection{Forms on Inner Product Spaces}

\begin{definition}
    A sesquilinear form on a vector space $V$ over $\mathbb{F}$ is a function $f: V \times
    V \rightarrow \mathbb{F}$ such that
    \begin{eqnarray*}
        (a) \quad f(c\alpha + \beta, \gamma) = cf(\alpha, \gamma) + f(\beta, \gamma) \\
        (b) \quad f(\alpha, c\beta + \gamma) = \overline{c}f(\alpha, \beta) + f(\alpha, \gamma)
    \end{eqnarray*}
\end{definition}

\begin{theorem}
    (p. 320)
    Let V be a finite dimensional inner product space and $f$ a form on $V$. Then
    \begin{enumerate}
        \item If $f \in \mathcal{F}(V,V,\mathbb{F})$, then there exists a unique linear
            operator $T_f: V \rightarrow V$ such that 
            \begin{equation*}
                f(\alpha, \beta) = \innerproduct{T_f \alpha, \beta} \quad 
                \forall \alpha, \beta \in V
            \end{equation*}
        \item The map $\phi: \mathcal{F}(V,V,\mathbb{F}) \rightarrow \mathcal{L}(V;V)$
            defined by $\phi(f) := T_f$ for all $f \in \mathcal{F}(V,V,\mathbb{F})$ is an
            isomorphism of vector spaces
    \end{enumerate}
\end{theorem}

\begin{definition}
    If $f$ is a form and $\mathcal{B} = \{\alpha_1, ..., \alpha_n\}$ an arbitrary ordered
    basis of V, the matrix A with entries
        \begin{equation*}
            A_{ij} = f(\alpha_j, \alpha_i)
        \end{equation*}
    is called the matrix of $f$ in the ordered basis of $\mathcal{B}$
\end{definition}

\begin{proposition} let $V$ be a finite dimensional inner product space over field $\mathbb{F}$
    and $f$ is a form. If $\mathcal{B}$ is an orthonormal basis of $V$, then $[f]_\mathcal{B}
    = [T_f]_\mathcal{B}$. \\
    (note that this lemma does not work if basis is not orthonormal)
\end{proposition}

\begin{theorem}
    Let f be a form on a finite dimensional complex inner product space V. Then there is
    an orthonormal basis for $V$ in which the matrix is upper-triangular. 

    (note that this theorem uses the fact that $\mathcal{F}(V,V,\mathbb{F})$ is isomorphic
    to $\mathcal{L}(V;V)$, which allows us to transfer the result of lemma \ref{lem3})
\end{theorem}


\begin{definition}
    A form $f$ on a real or complex vector space $V$ is called \textbf{Hermitian} if 
    \begin{equation*}
        f(\alpha,\beta) = \overline{f(\beta, \alpha)}
    \end{equation*}
\end{definition}

\begin{proposition}
    Let $f$ be a form on complex inner product space. $f$ is Hermitian if and only if $T_f$ is
    self-adjoint.
\end{proposition}

\begin{theorem}
    Let $V$ be a complex vector space and $f$ a form on $V$ such that $f(\alpha, \alpha)$
    is real for every $\alpha \in V$. Then $f$ is Hermitian.
\end{theorem}

\begin{corollary}
    Let $T$ be a linear operator on a complex finite dimensional inner product space $V$.
    Then $T$ is self-adjoint if and only if $\innerproduct{T\alpha, \alpha}$ is real for
    every $\alpha \in V$.
\end{corollary}

\begin{theorem}
    \textbf{Principle Axis Theorem: }
    For every Hermitian form $f$ on a finite dimensional inner product space $V$, there is
    an orthonormal basis in which $f$ is represented by a diagonal matrix with real
    entries.

    (Proof of this theorem uses results from theorem \ref{orthgonallydiagonalizable} and
    \ref{selfadjointrealandorthogonal})
\end{theorem}

\pagebreak

\section{Bilinear Forms}
Let $L(V,V;\mathbb{F})$ denote the space of all bilinear forms.

\subsection{Bilinear Forms Introduction}

\begin{definition}
    let V be a vector space over the field $F$. A bilinear form on $V$ is a function $f$,
    which assigns to each ordered pair of vectors $\alpha, \beta$ in $V$ a scalar
    $f(\alpha,\beta)$ in $F$, and which satisfies
    \begin{eqnarray*}
        (a) \quad f(c\alpha + \beta, \gamma) = cf(\alpha, \gamma) + f(\beta, \gamma) \\
        (b) \quad f(\alpha, c\beta + \gamma) = cf(\alpha, \beta) + f(\alpha, \gamma)
    \end{eqnarray*}
\end{definition}

\begin{definition}
    let $V$ be a finite-dimensional vector space and let $\mathcal{B}
    = {\alpha_1,...,\alpha_2}$ be an ordered basis for $V$. If $f$ is a bilinear form on
    $V$, the \textbf{matrix of $f$ in the ordered basis} $\mathcal{B}$ is the $n \times n$
    matrix $A$ with entries $A_{ij} = f(\alpha_i, \alpha_j)$. We will denote this matrix
    by $[f]_\mathcal{B}$
\end{definition}

\begin{theorem}
    let $V$ be a finite-dimensional vector space over the field $\mathbb{F}$. For each
    ordered basis $\mathcal{B}$ of $V$, the function which associates each bilinear form
    on $V$ its matrix in the ordered basis $\mathcal{B}$ is an isomorphism. i.e. The map
    \begin{equation*}
        \phi : L(V,V,\mathbb{F}) \rightarrow M_n(\mathbb{F})
    \end{equation*}
    is an isomorphism.
\end{theorem}

\begin{corollary}
    If $\mathcal{B} = \{\alpha_1,...\alpha_n\}$ is an ordered basis for $V$, and $B*
    = \{L_1,...,L_n\}$ is the dual basis for $V^*$, then the $n^2$ bilinear forms
        \begin{equation*}
            f_{ij}(\alpha, \beta) = L_i(\alpha)L_j(\beta)
        \end{equation*}
    form a basis for the space $L(V,V,\mathbb{F})$ and $dim(L(V,V,\mathbb{F})) = n^2$.
\end{corollary}
    
\begin{theorem}
    We denote $L_f(\alpha)$ as the bilinear form that fixes $\alpha$ hence becoming
    a linear functional on any $\beta$ in $V$. We denote $R_f(\beta)$ similarly but fixing
    $\beta$. 

    Let $f$ be a bilinear form on the finite-dimensional vector space $V$. Let $L_f$ and
    $R_f$ be the linear transformation from $V$ to $V^*$ defined by $(L_f\alpha)(\beta)
    = f(\alpha, \beta) = (R_f\beta)(\alpha)$. Then $\text{rank }(R_f) = \text{rank }(L_f)$.
\end{theorem}

\begin{definition}
    if $f$ is a bilinear form on the finite-dimensional space $V$, the \textbf{rank} of
    $f$ is the integer $r = \text{rank }(L_f) = \text{rank }(R_f)$.
\end{definition}

\begin{corollary}\label{rankbilinear}
    If $f$ is a bilinear form on the n-dimensional vector space $V$, the following are
    equivalent: 
    \begin{enumerate}
        \item rank $(f) = n$.
        \item For each non zero $\alpha$ in $V$, there is a $\beta$ in $V$ such that
            $f(\alpha, \beta) \neq 0$
        \item For each non zero $\beta$ in $V$, there is a $\alpha$ in $V$ such that
            $f(\alpha, \beta) \neq 0$
    \end{enumerate}
\end{corollary}

\begin{definition}
    A biliear form $f$ on a vector space $V$ is called \textbf{non-degenerate} (or
    \textbf{non-singular}) if it satisfies (2) and (3) of corollary \ref{rankbilinear}.
\end{definition}



\end{document}
