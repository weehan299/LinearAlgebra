\documentclass{article}
\usepackage[a4paper, inner=1.7cm, outer=2.7cm, top=2cm, bottom=2cm, bindingoffset=1.2cm]{geometry} 
\usepackage[english]{babel}
\usepackage{blindtext}
\usepackage{graphicx}
\usepackage{wrapfig}
\usepackage{enumitem}
\usepackage{amsmath}
\usepackage{index}
\usepackage{import}
\usepackage{epigraph}

\usepackage[bookmarks=true]{hyperref}
\usepackage{bookmark}

\makeindex

%new commands: 
\newcommand{\norm}[1]{\left\lVert#1\right\rVert}
\newcommand{\innerproduct}[1]{\langle#1\rangle}

\newtheorem{theorem}{Theorem}[section]
\newtheorem{corollary}{Corollary}[theorem]
\newtheorem{lemma}[theorem]{Lemma}
\numberwithin{theorem}{subsection} % important bit

\begin{document}
\title{\Large{\textbf{Linear Algebra: Map of theorems}}}
\author{Tan Wee Han}
\maketitle

\let\cleardoublepage\clearpage
\tableofcontents

\pagestyle{plain} %remove page headers but keep page numbers

\makeatletter
%\setlength{\@fptop}{0pt} % to put the figure on top of the page.
\makeatother


\section{Bilinear Forms and Inner Product}


\subsection{Vector inequalities}
Cauchy-Schwarz Inequality: $\norm{\innerproduct{u,v}} \leq \norm{u}\norm{v}$ \\
Triangle Inequality: $\norm{u+v} \leq \norm{u} + \norm{v}$ \\
Pythagoras Theorem: $\norm{u+v}^2 = \norm{u}^2 + \norm{v}^2$ \bigskip

\subsection{Orthogonal and Orthonormal basis}
\begin{theorem}
    An orthogonal set of nonzero vectors is linearly independent.
\end{theorem}

\begin{corollary}
    If $V$ is a finite dimensional inner product space and $n=dim V$, then any orthogonal set of nonzero vectors in $V$ is finite and contains at most n vectors.
\end{corollary}

\begin{lemma}
    Let $ \mathcal{B} = {v_1,...,v_n}$ be an orthonormal basis of $V$. Then for any $v \in V,$ \\
    \begin{equation*}
        V = \innerproduct{v,v_1}v_1 + \innerproduct{v,v_2}v_2 +...+
        \innerproduct{v,v_n}v_n
    \end{equation*}
\end{lemma}

\begin{corollary}

    Let $ \mathcal{B} = {v_1,...,v_n}$ be an orthonormal basis of $V$,
    $T:V\rightarrow V$ a linear operator and $[T]_\mathcal{B} = (a_{ij})$ Then for all $i,j$ \\
    \begin{equation*}
    a_{ij}=\innerproduct{T(v_j), v_i}
    \end{equation*}
\end{corollary}

\begin{theorem}
    Every finite dimensional inner product space has an orthonormal basis.
\end{theorem}



\subsection{Orthogonal complement and projection}
\begin{theorem}
    If $W$ is a subspace of an inner product space V, then its orthogonal complement
    $W^{\perp}$ is a subspace of $V$. In addition, we have
    \begin{equation*}
        W \cap W^{\perp} = \{\mathbf{0}\}
    \end{equation*}
\end{theorem}

\begin{theorem}
    If $W$ is a finite dimensional subspace of an inner product space V, then
    \begin{equation*}
        V = W \oplus W^{\perp} 
    \end{equation*}
\end{theorem}

\begin{theorem}
    If $\{w_1, ..., w_k\}$ is an orthonormal basis of $W$ then 
    \begin{equation*}
        \mathbf{proj}_W(v) = \Sigma^k_{j=1} \innerproduct{v,w_j}w_j
    \end{equation*}
\end{theorem}


\begin{theorem} 
    \textbf{Best Approximation:}
    If $W$ a finite dimensional subspace of an inner product space $V$ and $v \in V$, then
    \begin{equation*}
        \norm{v-\mathbf{proj}_W(v)} < \norm{v-w}
    \end{equation*}
    for every vector $w$ in $W$ different from $\mathbf{proj}_W(v)$.
\end{theorem}



\begin{theorem}
    \textbf{Least Square Solution:}
    For any real linear system $A\mathbf{x} = \mathbf{b}$ the associated normal system 
    \begin{equation*}
        (A^tA)\mathbf{x} = A^t \mathbf{b}
    \end{equation*}
is consistent, and all its solutions are least square solutions of $A\mathbf{x} = \mathbf{b}$
\end{theorem}


\subsection{Adjoint of linear operator}




\end{document}
